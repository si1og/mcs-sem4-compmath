\section*{Введение}
\addcontentsline{toc}{section}{Введение}

В отчёте представлено описание лабораторной работы №2 по <<Дискретной математике>>. Задана булева функция 4-х переменных (порядок определён по возрастанию элементов).

Цель работы~--- изучить булеву функцию, заданную десятичным числом, описав её математически и реализовав программно основные операции: построение нормальных форм, полинома Жегалкина и вычисление по бинарной диаграмме решений.

\vspace{0.5cm}

Задачи лабораторной работы:

\begin{enumerate}
    \item Построить таблицу истинности данной функции.
    \item Построить семантическое дерево решений.
    \item Построить наиболее компактную бинарную диаграмму решений, обосновать выбор порядка переменных.
    \item Записать формулу в виде ДНФ по БДР.
    \item Построить синтаксическое дерево по ДНФ.
    \item Вычислить первые производные по всем переменным.
    \item Вычислить четвёртую производную функции.
    \item Найти минимальную ДНФ и построить логическую схему.
    \item Программно построить СДНФ и СКНФ булевой функции.
    \item Программно реализовать хранение БДР и вычисление по ней.
    \item Программно построить полином Жегалкина и реализовать вычисление по нему.
\end{enumerate}

Таблица истинности задаётся значением функции $f = 11011_{10}$.